\documentclass[a4paper,11pt]{article}
\usepackage{graphicx}
\usepackage{times}
\usepackage[utf8]{inputenc} 
\usepackage[norsk]{babel} 
\topmargin=5.mm
\oddsidemargin=0.mm
\evensidemargin=0.mm
\textheight=220.mm 
\textwidth=170.mm
\parindent 0pt
\parskip .5em
\renewcommand{\arraystretch}{1.5}

\begin{document}
\sffamily
\begin{titlepage}
\begin{center}
\textsc{NORGES TEKNISK-NATURVITENSKAPELIGE UNIVERSITET\\
FAKULTET FOR INFORMASJONSTEKNOLOGI, MATEMATIKK OG ELEKTROTEKNIKK} \\
\vspace{0.5cm} 
\includegraphics[scale=0.5]{NTNU-logo} \\
\vspace{1.0cm}
{\Huge{PROSJEKTOPPGAVE}}
\vspace{1.0cm}
\end{center}

\begin{tabular}{@{}p{5cm}l}
Kandidatens navn:	& Haakon Garseg Mørk\\
Emne:			& Master thesis\\
Oppgavens tittel: 	& Security Properties of Information-centric Networks \\
Oppgavens beskrivelse: 	& \\
\end{tabular}

The translation from name to address and location is a fundamental problem to all networks.
Named Data Networking (NDN) is a proposal for content-centric discovery and routing approach to networking
going on at the University of California, Los Angeles (UCLA) [1], which is part of the inspiration and a contact point for this work.

This project aims to understand the ideas and concepts of named data networking, and investigate the
potential these notions hold for network security, even for properties of anonymity and privacy. 
The student work should be to build a public key distribution application that uses ChronoSync [2] and runs over NDN.
This approach will allow for easy experimentation in the regular internet protocols environment.     

[1]  UCLA.  Named Data Networking. Web site at http://named-data.net/

[2]  UCLA, Named-data Networking. ChronoSync https://github.com/named-data/ChronoSync

\begin{tabular}{@{}p{5cm}l}
Utf\o{}rt ved:	& Institutt for telematikk \\
Professor/veileder:	& Stig F. Mjølsnes \\
\end{tabular}

\end{titlepage}
\end{document}
