\documentclass[a4paper,11pt]{article}
\usepackage{graphicx}
\usepackage{times}
\usepackage[utf8]{inputenc} 
\usepackage[norsk]{babel} 
\topmargin=5.mm
\oddsidemargin=0.mm
\evensidemargin=0.mm
\textheight=220.mm 
\textwidth=170.mm
\parindent 0pt
\parskip .5em
\renewcommand{\arraystretch}{1.5}

\begin{document}
\sffamily
\begin{titlepage}
\begin{center}
\textsc{NORGES TEKNISK-NATURVITENSKAPELIGE UNIVERSITET\\
FAKULTET FOR INFORMASJONSTEKNOLOGI, MATEMATIKK OG ELEKTROTEKNIKK} \\
\vspace{0.5cm} 
\includegraphics[scale=0.5]{NTNU-logo} \\
\vspace{1.0cm}
{\Huge{PROSJEKTOPPGAVE}}
\vspace{1.0cm}
\end{center}

\begin{tabular}{@{}p{5cm}l}
Kandidatens navn:	& Haakon Garseg Mørk\\
Emne:			& Master thesis\\
Oppgavens tittel: 	& Security Properties of Information-centric Networks \\
Oppgavens beskrivelse: 	& \\
\end{tabular}

The translation from name to address and location is a fundamental problem to all networks.
Named Data Networking (NDN) is a proposal for content-centric discovery and routing approach to networking
going on at the University of California, Los Angeles (UCLA) [1], which is part of the inspiration and a contact point for this work.

In general, the name to address resolution can either be maintained by a catalogue lookup service, 
such as DNS (internet) and HLR (mobile networks), or resolved on-the-fly by a protocol on request, such as ARP (LAN). 
There has been a tremendous amount of work done on the naming problem in distributed systems, 
some became big failures (e.g. X.500) others such as the web URLs are very successful. Bringing things even further,
the Digital Object Identifier system [2] is a Uniform Resource Identifier (URI) directed at the content/object itself rather than a location. 
Very much related to the name/address problem is the information security problem of efficient and practical public key distribution, 
which remain unsolved in practice, 
even though a significant number of digital certificate and verification protocols and schemes have been proposed,
and systems tested over the last two decades. 
One notable and early theoretical proposal is Rivest and Lampson SDSI proposal [3],
and subsequent work, that may be revisited for applicable to named data networking.

This project aims to understand the ideas and concepts of named data networking, and investigate the
potential these notions hold for network security, even for properties of anonymity and privacy. 
It should be investigated whether the recent proposal by Håkon Wium Lie to use the Norwegian top domains of .sj and .bv as
privacy-friendly internet domains can be put in this context [4]. 

One idea for the student work can be to build a public key distribution application that uses ChronoSync [5] and runs over NDN.
This approach will allow for easy experimentation in the regular internet protocols environment.     

[1]  UCLA.  Named Data Networking. Web site at http://named-data.net/

[2]  The DOI system.  www.doi.org

[3]  Rivest, Ronald L., and Butler Lampson. SDSI-a simple distributed security infrastructure. Crypto, 1996.

[4]  Håkon Wium Lie.  En digital Svalbard-traktat. Dagbladet-kronikk 24.juni 2014.  

[5]  UCLA, Named-data. ChronoSync. Source code: https://github.com/named-data/ChronoSync

\begin{tabular}{@{}p{5cm}l}
Utf\o{}rt ved:	& Institutt for telematikk \\
Professor/veileder:	& Stig F. Mjølsnes \\
\end{tabular}

\end{titlepage}
\end{document}
