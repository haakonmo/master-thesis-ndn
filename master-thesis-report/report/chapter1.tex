\chapter{Introduction}\label{chp:introduction} 

\section{Motivation}
The translation from name to address and location is a fundamental problem to all networks.
\gls{NDN} is a proposal for content-centric discovery and routing approach to networking
going on at the \gls{UCLA}, which is part of the inspiration and a contact point for this work.

In general, the name to address resolution can either be maintained by a catalogue lookup service, 
such as \gls{DNS} (internet) and \gls{HLR} (mobile networks), 
or resolved on-the-fly by a protocol on request, such as \gls{ARP} (\gls{LAN}). 
There has been a tremendous amount of work done on the naming problem in distributed systems, 
some became big failures (e.g. X.500) others such as the web \gls{URL}s are very successful. 
Bringing things even further, the \gls{DOI} system is a \gls{URI} directed at the content/object itself rather than a location. 
Very much related to the name/address problem is the information security problem of efficient and practical public key distribution, 
which remain unsolved in practice, even though a significant number of digital certificate and verification protocols and schemes have been proposed, and systems tested over the last two decades. 
One notable and early theoretical proposal is Rivest and Lampson SDSI proposal~\cite{rivest1996sdsi},
and subsequent work, that may be revisited for applicable to named data networking.

\section{Problem and Scope}


\section{Methodology}


\section{Outline}

