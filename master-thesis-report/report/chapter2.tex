\chapter{Background}\label{chp:background} 
\chapterprecishere{"We model the future on the past. Sometimes that’s a mistake."\par\raggedleft--- \textup{Van Jacobsen}, SIGCOMM 2001}

This chapter will give a brief overlook of the motivation for \gls{ICN},
as well as explaining more details of how \gls{ICN} protocols, such as \gls{CCN} and \gls{NDN}, works. 
It will also contain a quick summary of related works.

\section{Motivation for \gls{ICN}}
IP/TCP not designed for broadcast
\todo{Find sources for content based traffic}
Not designed for content, but rather point-to-point. 
Why? Telecommunication architecture. 
Natural and logical to design after these principles.
Security? Not designed with security in mind. IPSec
Nodes know nothing

Motivation for \gls{ICN}
\gls{IRTF} established \gls{ICN} working group 2012

\section{\gls{ICN} Protocols}\label{chp2:sec:icn}
\gls{CCN} started in 2006
\gls{PARC}
Publicly presented by Van Jacobsen in 2006
Jacobsen has contributed to TCP/IP, i.a. with his flow control algorithms and TCP header compression. 

\gls{NDN}~\cite{DBLP:journals/ccr/0001ABJcCPWZ14} started in 201*\todo{what year? 2012?}
\gls{UCLA}
Lixia Zhang

\section{NDN Architecture}\label{chp2:sec:ndn_architecture}
\gls{NDN} ...

\todo{network illustation}
\todo{node illustation}

Describe the architecture ~\cite{NDN-0021}

Interest packet
Data packet

Pull-based which means that there is no content in the network, that is not requested from someone.
Reduces unwanted traffic/content. 

Security aspects of the network protocol. 
Demands security from application layer.

Nodes know WHAT (name) they have. 
Nodes know what content they have, hence can share relevant data.

Face - what is a face? Ethernet, TCP/UDP, Websocket
strip away link layer headers (if any)

\gls{PIT} - All pending or recently satisfied interests are stored here.


\gls{CS} - Cache of data packets. If match here, send back data packet, otherwise for Interest

\gls{FIB} - 
\gls{RIB} - 
 
\subsection{Incoming Interest}
Incoming Interest through a Face. Check \gls{PIT} for pending or recently satisfied interests. If no match, lookup in \gls{CS}s

\subsection{Outgoing Interest}
How to forward a Interest; routing strategy. A strategy per \gls{PIT} entry. I.e. whether, when , and where to forward Interest.

\subsection{Incoming Data}
--> check \gls{PIT}, if match., check data from local applications first cached in \gls{CS}, if not, store in \gls{CS} send data to all requesters (\gls{PIT})
Send data via Face

\subsection{Outgoing Data}


\subsection{Naming}
Naming is left to the application design.

\gls{DNS} naming
Email naming
\gls{OS} naming

\subsection{Anonymity}
Describe the concept, based on the nature of the architecture.

Tor network

\section{Attacks}

\subsection{\gls{DoS} attack}
DDoS~\cite{DBLP:conf/icccn/GastiTU013}
Summary:


modeling ddos~\cite{DBLP:journals/ijcomsys/WangCZQZ14}
Summary:


DDoS-resistant forwarding with hash tables~\cite{DBLP:conf/ancs/SoNO13}
Summary:


Mitigating Interest Flooding DDoS Attacks~\cite{DBLP:journals/corr/abs-1303-4823}
Summary:

\subsection{Cache attack}

Lightweight Integrity Verification and Content Access Control for Named Data Networking~\cite{DBLP:journals/tifs/LiZZSF15}
Summary:


Network-Layer Trust in Named-Data Networking~\cite{DBLP:journals/ccr/GhaliTU14}
Summary:


\section{Related work}
iSync~\cite{DBLP:conf/acmicn/FuAC14}
Summary:


Reliable retrieval of data from different wireless producers which can answer to the same Interest packet~\cite{DBLP:conf/acmicn/AmadeoCM14}
Summary:


Named data aggregation in wireless sensor networks~\cite{DBLP:conf/noms/AbidySLF14}
Summary:


Secure Sensing over Named Data Networking~\cite{DBLP:conf/nca/BurkeGNT14}
Summary:

