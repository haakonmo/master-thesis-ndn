\chapter{Background}\label{chp:background} 
\chapterprecishere{"We model the future on the past. Sometimes that’s a mistake."\par\raggedleft--- \textup{Van Jacobsen}, SIGCOMM 2001}

This chapter will give a brief overlook of the motivation for \gls{ICN},
as well as explaining more details of how \gls{ICN} protocols, such as \gls{CCN} and \gls{NDN}, works. 
It will also contain a quick summary of related works.

\section{Motivation for \gls{ICN}}
When Internet was created in the 1960's, the researchers where inspired by the existing communication network, the telecommunication network.
Because it was not natural and logical to think that people would download content, but rather send and receive short messages and instructions, the node to node communication model was a logical architecture. 
As internet have developed, we now know that close to 80\% \todo{find cite} of all \gls{IP} traffic is content download, and not communication. 
With this in mind, the \gls{IP} architecture does not provide an effictient transport model for what we are actually using the network for.

The \gls{IP} network was not designed with security in mind. 
Most of the protocols related to Internet have been designed and deployed mainly with the goal of functionality to work.
\gls{IPsec} is a very good example. 

A node in a \gls{IP} network does not know WHAT goes through, but rather the packet's endpoints, i.e. WHERE it goes and WHERE it comes from. 
This makes every node dumb, hence the network is designed for redundancy.

IP/TCP not designed for broadcast. 
Wireless connection is not easy.
\todo{Find sources for content based traffic}

These concepts are the reason why we need \gls{ICN}. 
\gls{IRTF} established \gls{ICN} working group 2012

\section{\gls{ICN} Protocols}\label{chp2:sec:icn}
\gls{CCN} was publicly presented by Van Jacobsen in 2006.
The research project is lead by \gls{PARC}.
Jacobsen has contributed to TCP/IP, i.a. with his flow control algorithms and TCP header compression. 

A branch of \gls{CCN} is the \gls{NDN}~\cite{DBLP:journals/ccr/0001ABJcCPWZ14} research project started in 2012\todo{what year? 2012?}
One of the biggest contributers is \gls{UCLA}, with Lixia Zhang in the lead.

\section{NDN Architecture}\label{chp2:sec:ndn_architecture}
\gls{NDN} ...

\todo{network illustation}

\subsection{Node}
\todo{draw own figure}
\begin{figure}[ht]
  \centering
  \includegraphics[width=1\textwidth]{ip_model_node.png}
  \caption{\gls{IP} model}
  \label{fig:ip-model-node}
\end{figure}

\todo{draw own figure}
\begin{figure}[ht]
  \centering
  \includegraphics[width=1\textwidth]{icn_model_node.png}
  \caption{\gls{ICN} model }
  \label{fig:icn-model-node}
\end{figure}

Describe the architecture ~\cite{NDN-0021}

Pull-based which means that there is no content in the network, that is not requested from someone.
Reduces unwanted traffic/content. 
Two types of packets in \gls{NDN}. Interest packet and the corresponding answer, i.e. data packet, illustrated in~\autoref{fig:packets}.

\todo{draw own figure}
\begin{figure}[ht]
  \centering
  \includegraphics[width=1\textwidth]{packets.png}
  \caption{Packets}
  \label{fig:packets}
\end{figure}

Security aspects of the network protocol. 
Demands security from application layer.

Nodes know WHAT (name) they have. 
Nodes know what content they have, hence can share relevant data.

Face - what is a face? Ethernet, TCP/UDP, Websocket
strip away link layer headers (if any)

\subsubsection{\gls{PIT}}
\gls{PIT} - All pending or recently satisfied interests are stored here.
~\autoref{fig:icn-model-node}

\subsubsection{\gls{CS}}
\gls{CS} - Cache of data packets. If match here, send back data packet, otherwise for Interest
~\autoref{fig:icn-model-node}

\subsubsection{\gls{FIB}}
\gls{FIB} - 
~\autoref{fig:icn-model-node}

\subsubsection{\gls{RIB}}
\gls{RIB} - 
~\autoref{fig:icn-model-node}
 
\subsection{Incoming Interest}
Incoming Interest through a Face. Check \gls{PIT} for pending or recently satisfied interests. If no match, lookup in \gls{CS}s

\subsection{Outgoing Interest}
How to forward a Interest; routing strategy. A strategy per \gls{PIT} entry. I.e. whether, when , and where to forward Interest.

\subsection{Incoming Data}
--> check \gls{PIT}, if match., check data from local applications first cached in \gls{CS}, if not, store in \gls{CS} send data to all requesters (\gls{PIT})
Send data via Face

\subsection{Outgoing Data}


\subsection{Naming}
Naming is left to the application design.

\gls{DNS} naming
Email naming
\gls{OS} naming

\subsection{Anonymity}
Describe the concept, based on the nature of the architecture.

Tor network

\section{Attacks}

\subsection{\gls{DoS} attack}
DDoS~\cite{DBLP:conf/icccn/GastiTU013}
Summary:


modeling ddos~\cite{DBLP:journals/ijcomsys/WangCZQZ14}
Summary:


DDoS-resistant forwarding with hash tables~\cite{DBLP:conf/ancs/SoNO13}
Summary:


Mitigating Interest Flooding DDoS Attacks~\cite{DBLP:journals/corr/abs-1303-4823}
Summary:

\subsection{Cache attack}

Lightweight Integrity Verification and Content Access Control for Named Data Networking~\cite{DBLP:journals/tifs/LiZZSF15}
Summary:


Network-Layer Trust in Named-Data Networking~\cite{DBLP:journals/ccr/GhaliTU14}
Summary:


\section{Related work}
iSync~\cite{DBLP:conf/acmicn/FuAC14}
Summary:


Reliable retrieval of data from different wireless producers which can answer to the same Interest packet~\cite{DBLP:conf/acmicn/AmadeoCM14}
Summary:


Named data aggregation in wireless sensor networks~\cite{DBLP:conf/noms/AbidySLF14}
Summary:


Secure Sensing over Named Data Networking~\cite{DBLP:conf/nca/BurkeGNT14}
Summary:

