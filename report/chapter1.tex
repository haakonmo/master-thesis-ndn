\chapter{Introduction}\label{chp:introduction} 

\section{Motivation}
The translation from name to address and location is a fundamental problem to all networks.
\gls{NDN} is a proposal for content-centric discovery and routing approach to networking
going on at the \gls{UCLA}, which is part of the inspiration and a contact point for this work.

In general, the name to address resolution can either be maintained by a catalogue lookup service, 
such as \gls{DNS} (Internet) and \gls{HLR} (mobile networks), 
or resolved on-the-fly by a protocol on request, such as \gls{ARP} (\gls{LAN}). 
There has been done a tremendous amount of work on the naming problem in distributed systems, 
some became big failures (e.g. X.500) others such as the web \gls{URL}s are very successful. 
Bringing things even further, the \gls{DOI} system is a \gls{URI} directed at the content/object itself rather than a location. 
Very much related to the name/address problem is the information security problem of efficient and practical public key distribution, 
which remain unsolved in practice, even though a significant number of digital certificate and verification protocols and schemes have been proposed, and systems tested over the last two decades. 
One notable and early theoretical proposal is Adi Shamir's \gls{IBC} proposal~\cite{DBLP:conf/crypto/Shamir84},
and subsequent work, that may be revisited and applicable to \gls{NDN}.

\section{Problem and Scope}

Designing a new network protocol for the future Internet, one of the most significant changes should be security.
Trust management plays a big part in security, and thus we cannot design trust management on known \gls{IP} failures such as X.500. 
\gls{PKI} is a tough challenge to solve and it is probably not a rigid solution, but rather case specific.
\gls{NDN} is being designed with security in mind, but the issue of trust management is yet to be solved.

In addition to the problem description, I also address the trust management issue in a thought sensor device network using \gls{IBC}.
By using the \gls{NFD} I will implement my proposal for such sensor network over \gls{NDN}, and contribute with ideas and concepts around such network.

\section{Methodology}

When developing applications over \gls{NDN} it is important to understand the architecture of the protocol. 
I will study the protocol, reading the guide for NDN developers~\cite{NDN-0021} and papers published by the \gls{NDN}-team\footnote{NDN Publications - http://named-data.net/publications/} describing protocol features in addition to existing applications running over \gls{NDN}.

The concept of \gls{IBC} should be thoroughly understood, as I will apply it in my application.
Thus I will study \gls{IBC}, finding relevant papers at Google Scholar\footnote{Google Scholar - https://scholar.google.no/}.

Establishing a solid background knowledge in these topics, I first design the application flow in sequence diagrams.
Based on the \gls{api} to the \gls{PyNDN2}, I try to implement the proposed design and see where changes can be made to minimize communication overhead, maximize security (i.e. \gls{CIA}) and maximize usability.
The implementation will be tested and cryptographic parameters will be measured.
I will prove the security in the protocols I propose and finally I will discuss the work done.

\section{Outline}

This thesis will first introduce \gls{NDN}, one of the proposed protocols for the future Internet.
I will explain the architecture of \gls{NDN} as well as some related work regarding my application proposal and \gls{IBC}. 
The concept of \gls{IBC} will also be explained.
The specifications for the \gls{HSS} application will be explained in detail and implementation choices will be discussed.
I will present the results of the implementation and testing.
At last, a discussion around the research topics in the thesis will be presented and finally my conclusion around the same topics.
