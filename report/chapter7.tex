\chapter{Conclusion and Future Work}\label{chp7:conclusion}
In this chapter the conclusion of this thesis will be presented and the future work will be listed.

\section{Conclusion}
A new network protocol is much needed due to the lack of security in existing networks and the continually increase of data traffic around the world. 
This thesis explains the architecture of the proposed future Internet protocol \gls{NDN}.
\gls{NDN} facilitates a lot of concepts that shows to be a huge benefit for todays Internet, and the predicted increase of \gls{IoT}.
The naming of content and content routing provides usability to \gls{IoT} and \gls{wsn}.
The concept of \gls{IBC} shows to be highly applicable to \gls{IoT} and \gls{wsn}, and running \gls{IBC} over \gls{NDN}, makes it even more practical.
This is because of the content naming concept that \gls{NDN} is built upon. 
It is easier to secure data, relate data to publisher, and authenticate that the publisher is aware of what content it published, than in an \gls{IP} network. 
Using \gls{IBC} in a \gls{wsn} running over \gls{NDN} makes applications with security less complex and more practical than using equivalent security such as RSA. 

In this thesis I have developed an application thought to be deployed in a \gls{wsn}, written in Python with \gls{IBC} used for signing and verification, encryption and decryption.
The system is running over the new network protocol called Named Data Networking.
This work shows how applicable \gls{NDN} together with \gls{IBC} are for \gls{IoT}, and to design secure protocols for local device networks.
The application is tested to see how the suggested protocols for device registration and data pull performs with \gls{IBC}. 
I have proven that my suggested proposal is a secure system that can easily be implemented, achieving confidentiality, integrity and authenticity, as well as trust.

\section{Future Work}
The implementation of the \gls{HSS} does not include integration of the \gls{FSM}, which is a part of the future work.
The system is not tested on relevant sensors and devices to measure latency and performance.

An implementation of a full worthy \gls{IBC} solution in \gls{PyNDN2} is not implemented.
This implementation should include making \gls{IBC} as a part of the PyNDN2 framework, so that developers easily can make use of \gls{IBE} and \gls{IBS} performing encryption, decryption, signing and verification.

The \gls{IBC} schemes used in the Charm framework does not provide a scheme that implements \gls{IBE} and \gls{IBS} together in one scheme.
This should not be a huge task to implement, but it will decrease the device registration round-trip time as well as minimizing the use of several keys, i.e. easier key management.
It is explained how to derive a signature scheme from any \gls{IBE} scheme~\cite[Section 4]{DBLP:conf/crypto/Waters09}.
